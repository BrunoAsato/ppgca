\documentclass[12pt,a4paper]{article}
\usepackage[utf8]{inputenc}
\usepackage[brazil]{}
\usepackage[top=3cm,bottom=2cm,left=3cm,right=2cm]{geometry}


% Title Page
\title{Ficha de Leitura}
\author{Bruno Asato}

\begin{document}
% \maketitle

\textbf{Data de leitura:} 14/02/2019

\textbf{Título do artigo:} Efeito da diversidade funcional sobre o aproveitamento
da luz em sistemas agroflorestais sucessionais

\textbf{Autor(es):} Monteiro, alvaro

\textbf{Ano:} 2017

\textbf{Objeto de estudo:} Sistemas agroflorestais

\textbf{Questões/Hipóteres do autor:} A diversidade funcional influencia no desenvolvimento dos cultivares em um mesmo sistema.

\textbf{Palavras chaves:}

\textbf{Metodologia empregada:} Avaliação da RFA em linhas e entrelinhas de SAFs e desenvolvimento 

\textbf{Apresentação de resultados e principais resultados:}

\textbf{Conclusões:} A diversidade funcional com herbáceas e eretas mostraram-se imporantes como forma de crescimento a serem consideradas na constução de SAFs.

\textbf{Avaliação crítica:}



\end{document}          
